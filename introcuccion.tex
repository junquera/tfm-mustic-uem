\chapter{Introducción}

Si para todo $ A,B \in \chi{} $, una operación $ \bigoplus $, y una función $f$; si se cumple $ f(A) \bigoplus f(B) = f(A \bigoplus B)$, $ f $ es una función homomórfica con respecto a $ \bigoplus $ en $ \chi{} $. Cuando una función criptográfica cumple esta condición con alguna operación se dice que es maleable (\cite{dolev_non-malleable_1991}). 

Aunque es una propiedad que podría no ser deseable en la muchos ámbitos (por ejemplo, en un escenario en el que además de confidencialidad se requiera integridad, permitiría a un atacante modificar los datos), si cumple ciertas condiciones podría tener numerosas aplicaciones, creando el campo de estudio de la criptografía homomórfica.

Se conoce como criptografía homomórfica al conjunto de técnicas criptográficas destinadas a permitir operar con datos cifrados y que dichas operaciones se materialicen correctamente sobre los datos al descifrarlos. En función (principalmente) de las operaciones con las que se cumple esta premisa, o el sistema utilizado para procesar los datos antes y después de operar... Aunque pueda haber muchas variantes de cada una de estas propiedades, la comunidad científica ha establecido criterios y notaciones para su estudio. 

\section{Estandarización}

El consorcio "Homomorphic Encryption Standardization" (\cite{noauthor_homomorphic_nodate-1}) ha ido desarrollando un estándar a lo largo de los años atendiendo a los avances en las distintas tecnologías que componen la criptografía homomórfica, y prestando un interés especial en las implementaciones necesarias para ponerla en práctica. Así, se han ido sucediendo las tres generaciones de criptografía homomórfica.

Se está desarrollando el cuarto encuentro de estandarización.


\subsection{API}

OPERACIONES DEL ESTÁNDAR

\section{Aplicaciones}

APLICACIONES PRÁCTICAS: Medicina, computación distribuida de forma segura...

\section{Trabajo}

Para nuestro trabajo utilizaremos las librerías SEAL (como representante de la segunda generación) y THFE (como representante de la tercera).

Comenzamos con un estudio del funcionamiento de las librerías, de cuales son sus límites computacionales y pienso una operación que no sea "trivial" y a la vez sea computable por ambos.

La librería SEAL cuenta con varias operaciones aritméticas implementadas que pueden facilitarnos mucho el trabajo, así como la posibilidad de trabajar con matrices y número con coma flotante. La principal complicación con SEAL es calcular cómo se está desviando el cálculo para saber cuando parar o qué correcciones hacer. Además, las operaciones que puede hacer son relativamente limitadas (cuando se hacen, por ejemplo, productos, crece mucho el error). Si se ajustan los parámetros para realizar más operaciones o trabajar con números más grandes, se vuelve lentísima.

La librería TFHE ofrece una API de operaciones lógicas a nivel de bit. Aunque permitirá hacer cálculos más complejos sin añadir error al resultado, tendremos que implementar todas las operaciones a bajo nivel mediante puertas lógicas, siempre teniendo en cuenta que en ningún momento vamos a poder controlar el flujo de ejecución y nuestros algoritmos tienen que ser capaces de realizar los cálculos sin poder evaluar las variables (están cifradas), lo que hará que nuestras operaciones sean

Por lo tanto tenemos que encontrar un cálculo que sea realizable con pocas operaciones y números bajos para SEAL; y operaciones que sean implementables en un tiempo razonable con puertas lógicas para poder trabajar en TFHE.

Además, la operación que se realice con SEAL no puede contener más que sumas, restas y multiplicaciones (no existe la división).

He decicido hacer una regresión cuadrática de temperaturas de dos ciudades (esto lo haré con TFHE para poder hacer suma, resta, multiplicación y DIVISIÓN), que permitirá que un usuario en una fecha dada suba a SEAL su temperatura cifrada y averigue su localización.

Generaremos "el modelo" (la curva de regresión f(x)) con TFHE, porque para ello necesitaremos realizar operaciones muy costosas (elevando hasta la 4) y necesitamos dividir para hacer la media. Evaluaremos los datos introducidos por el usuario en SEAL (distancia entre f("x introducido por el usuario") y "y introducida por el usuario").