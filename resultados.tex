\chapter{Resultados}
\label{chap:resultados}

% TODO Inlcuir datos de las máquinas

\section{TFHE}

Al ser una ejecución "especulativa"...

\subsection{Tiempos de ejecución}

\subsection{Tamaño máximo de los datos}

% log(X, 2)*max_exponente <= (64 bits de entero - 10 bits de decimal - 1 bit de signo) = 53 bits

% 4 < max_exponente < 10

% X < 32

\subsection{Problemas encontrados}

- Signo

- Floats

- Eficiencia

% TODO Documentar tiempos

- Tamaño de los datos al multiplicar...

% TODO Mostrar algunos ejemplos de codificación de números en bits
l = sum(1, n)
nb_bits > 1 + math.log(n, 2) + 10

\section{Coste de la implantación}

Equipo de ingenieros, estudio, pruebas...

Máquina digital ocean con cálculo (tiempo*precio).

Los resultados para curva A han sido: ...
para curva B han sido: ...

\section{SEAL}

\subsection{Tiempos de ejecución}

\subsection{Límites de cómputo}

Con CKKS, por tamaño de la cadena:

El primer y el último número de la cadena tienen que ser mayores que el número a cifrar/descifrar, y los intermedios tienen que ser lo algo más grandes  que los intermedios para asegurar la precisión, y la suma de estos dos con los intermedios tiene que ser menos que  \verb|max coeff_modulus| bit-length. POr lo tanto:

\begin{itemize}
    \item Para un número de 40 bits, es necesario utilizr al menos  \verb|poly_modulus_degree de 8192|, y se pueden hacer 4 operaciones.
    \item Para un número de 64 bits, es necesario utilizr al menos  \verb|poly_modulus_degree de 16384|, y se pueden hacer 7 operaciones.
    \item Para un número de 64 bits, es necesario utilizr al menos \verb|poly_modulus_degree de 16384|, y se pueden hacer 14 operaciones.
\end{itemize}

Este es el número de operaciones tras el cual el resultado es inválido.

Con BFV sólo enteros, y por niveles de error:
