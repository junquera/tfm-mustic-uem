\chapter{Introducción}
\label{chap:intro}

Para todo $ A,B \in \chi{} $, una operación $ \bigoplus $, y una función $f$; si se cumple $ f(A) \bigoplus f(B) = f(A \bigoplus B) $, $ f $ es una función homomórfica con respecto a $ \bigoplus $ en $ \chi{} $. Cuando una función criptográfica cumple esta condición con alguna operación se dice que es maleable (\cite{dolev_non-malleable_1991}).

Aunque es una propiedad que podría no ser deseable en muchos ámbitos (por ejemplo, en un escenario en el que además de confidencialidad se requiera integridad, permitiría a un atacante modificar los datos), si cumple ciertas condiciones puede tener numerosas aplicaciones. Así se crea el campo de estudio de la criptografía homomórfica.

Se conoce como criptografía homomórfica al conjunto de técnicas criptográficas destinadas a permitir operar con datos cifrados y que dichas operaciones se materialicen correctamente sobre los datos al descifrarlos. En función (principalmente) de las operaciones con las que se cumple esta premisa, o el sistema utilizado para procesar los datos antes y después de operar, los esquemas con propiedades homomórficas se categorizarán de una forma u otra. Aunque pueda haber muchas variantes de cada una de estas propiedades, la comunidad científica ha establecido criterios y notaciones para su estudio.

\section{Estandarización}

El consorcio "Homomorphic Encryption Standardization" (\cite{albrecht_homomorphic_2018}) ha ido desarrollando un estándar a lo largo de los años atendiendo a los avances en las distintas tecnologías que componen la criptografía homomórfica, y prestando un interés especial en las implementaciones necesarias para ponerla en práctica. Así, se han ido sucediendo las tres generaciones de criptografía homomórfica.

El último encuentro de trabajo del grupo se produjo el 17 de Agosto en Santa Clara, y en él se trabajará principalmente la eficiencia, la seguridad y la usabilidad de las librerías.

La documentación del estándar está dividida en tres \textit{white papers} y una lista de implementaciones conocidas.

\subsection{Seguridad}

En el documento (\cite{chase_security_2017}) analizan qué principios seguir para implementar esquemas de criptografía homomórfica y qué beneficios tienen estos esquemas para la seguridad de la información. También hacen un análisis de los ataques existentes a estos esquemas y qué parámetros matemáticos son los ideales para hacer estos esquemas lo más resistentes posible tanto en el panorama actual como en un escenario \textit{post-quantum}.

\subsection{Aplicaciones}

En el estudio (\cite{archer_applications_2017}) abordan para qué campos es útil la criptografía homomórfica. Mientras que se han estado centrando los esfuerzos en poder almacenar información en la nube de forma segura (cifrándola) se ha descuidado la seguridad de dicha información cuando sube a la nube para ser procesada. La criptografía homomórfica puede ser la solución a este problema, y en campos como:

\begin{itemize}
    \item La computación distribuida
    \item La protección de datos médicos que tengan que ser procesados en equipos potentes, ajenos a la institución médica
    \item La consulta de información de forma anónima: por ejemplo, una consulta DNS sin revelar a qué se está accediendo
\end{itemize}

En la última reunión del standard presentaron ejemplos de protocolos de distribución de datos y claves (\cite{troncoso-pastoriza_homomorphic_nodate}) utilizando nuevas implementaciones como Lattigo (\cite{noauthor_lattigo_2019}).

\subsection{API}

Por último el consorcio de estandarización busca establecer un modelo de almacenamiento común tanto de los datos como de las claves, y una terminología (lo llaman \textit{lenguaje ensamblador}) que sirva de lenguaje común para transmitir las ideas y los elementos mínimos que debe tener cualquier sistema de criptografía homomórfica. Están definidos en el documento (\cite{brenner_standard_2017}), y como veremos más adelante, encajan perfectamente con los elementos de las implementaciones. Son los siguientes:

\begin{itemize}
    \item \verb|SecKeygen|, \verb|PubKeygen|

    Tiene que haber un método de generación de clave privada (o secreta) y pública.

    \item \verb|SecEncrypt|, \verb|PubEncrypt|

    Define la posibilidad de que haya además de un sistema de cifrado público (usando la clave pública) que en determinados esquemas se pueda cifrar la información directamente con la clave privada.

    \item \verb|Decrypt|

    Tiene que haber un sistema para restaurar la información desde un texto cifrado.

    \item \verb|Eval|

    La llamada \verb|Eval| será el paraguas que recoja todas las operaciones homomórficas que se puedan realizar sobre el texto cifrado para que luego se materialicen en el descifrado.

\end{itemize}

Algunos esquemas introducirán otras herramientas enfocadas a procesar los datos, pero estarían en niveles superiores de la arquitectura, y son exclusivas de cada implementación.

\section{Objetivo del trabajo}

El objetivo de este trabajo es evaluar si es viable o no utilizar las tecnologías existentes en sistemas y procesos reales. Estudiaremos qué implementaciones hay, en qué consisten, y cuales serán las más idóneas (las más avanzadas, que puedan servir de muestra para inferir la viabilidad de las demás) atendiendo a:

\begin{itemize}
    \item La facilidad de uso: A fin de cuentas la documentación existente y la mayor o menor facilidad de uso se traduce en horas de salario de trabajadores altamente cualificados.
    \item Las capacidades de la tecnología: O qué problemas pueden resolverse con ella
    \item La eficiencia de la solución: Estudiar los tiempos de ejecución de las operaciones
\end{itemize}
