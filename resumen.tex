\chapter*{Resumen}
\label{chap:resumen}

% Máximo 200 palabras
Se conoce como criptografía homomórfica al conjunto de técnicas destinadas a cifrar los datos de tal forma que las operaciones que se apliquen sobre el texto cifrado se manifiesten en el texto plano al descifrar. En este trabajo analizaremos las distintas técnicas existentes para este propósito, cuáles son sus bases teóricas, y evaluaremos si es viable o no utilizar las implementaciones disponibles en sistemas destinados al uso en producción. 

Las principales implementaciones de criptografía homomórfica protegen el texto utilizando un esquema de cifrado llamado \textit{Learning With Errors}, y estarán categorizadas como \textit{Partially Homomorphic Encryption}, \textit{Somewhat Homomorphic Encryption} (SHE) y \textit{Fully Homomorphic Encryption} (FHE) en función de las propiedades homomórficas que cumplan. Para nuestra evaluación construiremos un sistema que utilice una implementación de tipo SHE y otra de tipo FHE, y compararemos los resultados desde el punto de vista de la eficiencia, de la capacidad de cómputo y de la facilidad de desarrollo. 

Concluiremos mostrando cómo existe cierta viabilidad desde el punto de vista tecnológico a la hora de utilizar sistemas de criptografía homomórfica, pero la viabilidad económica dependerá tanto del valor del activo como del riesgo que se busque mitigar, pues todavía no existe una solución universal.