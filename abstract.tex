\chapter*{Abstract}
\label{chap:abstract}

Homomorphic Encryption is the set of techniques designed to encrypt data so that an operation performed over encrypted data remains in the data when decryption. In this work, we will analyze the different existing techniques for this purpose, what are its theoretical bases, and we will evaluate the viability for using the different existing implementations in production systems.

The main homomorphic encryption implementations protect the text using a schema known as \textit{Learning With Errors} and are categorized as \textit{Partially Homomorphic Encryption}, \textit{Somewhat Homomorphic Encryption} (SHE) and \textit{Fully Homomorphic Encryption} (FHE) depending on the homomorphic encryption properties they fulfill. For our implementation, we will build a system using both a SHE, and an FHE implementation, and we will compare the results from the point of view of the efficiency, the computing capacity and the easiness of development.

We will conclude showing how it exists certain viability from the point of view of the technology when using homomorphic encryption systems, but the economic viability depends on both the active value and the risk to be mitigated, as there is still no a universal solution.